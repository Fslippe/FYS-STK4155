\documentclass[norsk,a4paper,12pt]{scrartcl}
\usepackage[utf8]{inputenc}
\usepackage{amsmath}
\usepackage{graphicx}
\usepackage{listings}
\usepackage{verbatim}
\usepackage{float}
\usepackage{xcolor}
\usepackage{subcaption}
\usepackage{caption}
\usepackage{hyperref}
%\usepackage[a4paper, total={6in, 9in}]{geometry}
\renewcommand{\figurename}{Figur}
\renewcommand{\contentsname}{Innhold}
\hypersetup{
    colorlinks,
    linkcolor={red!50!black},
    citecolor={blue!50!black},
    urlcolor={blue!80!black}
}
\definecolor{codegreen}{rgb}{0,0.6,0}
\definecolor{codegray}{rgb}{0.5,0.5,0.5}
\definecolor{codepurple}{rgb}{0.58,0,0.82}
\definecolor{backcolour}{rgb}{0.95,0.95,0.92}
\definecolor{dkgreen}{rgb}{0,0.6,0}
\definecolor{gray}{rgb}{0.5,0.5,0.5}
\definecolor{mauve}{rgb}{0.58,0,0.82}

\lstdefinestyle{mystyle}{
    backgroundcolor=\color{backcolour},
    commentstyle=\color{codegreen},
    keywordstyle=\color{magenta},
    numberstyle=\tiny\color{codegray},
    stringstyle=\color{codepurple},
    basicstyle=\ttfamily\footnotesize,
    breakatwhitespace=false,
    breaklines=true,
    captionpos=b,
    keepspaces=true,
    numbers=left,
    numbersep=5pt,
    showspaces=false,
    showstringspaces=false,
    showtabs=false,
    tabsize=2
}

\lstset{style=mystyle}
\title{}

\author{}
\date{\today}
\begin{document}
\maketitle

\section{}
By assuming that the continous funcitiion $f(x)$ can be parametrized in terms of a polynomial of degree $n-1$ means that our datapoints can be described by the following expression:
\begin{align*}
  \boldsymbol{y} = \sum_{j=0}^{n-1}\beta_jx_i^j + \epsilon_i
\end{align*}
This is the same as the following matrix equation:
\begin{align*}
  \boldsymbol{y} = \boldsymbol{X}\boldsymbol{\beta} + \boldsymbol{\epsilon} = \boldsymbol{\tilde{y}} + \boldsymbol{\epsilon}
\end{align*}
This means that the elemnt $i$ is given by:
\begin{align*}
  y_i = \epsilon_i + \sum_j x_{ij}\beta_j
\end{align*}
The expactation value of the element $i$ in $\boldsymbol{y}$:
\begin{align*}
  E(y_i) = E(\epsilon_i + \sum_j x_{ij}\beta_j)
\end{align*}
Since we already know that $\boldsymbol{\epsilon}$ is normal distibuted with the expectation value 0 $E(\epsilon_i)=0$. This gives us:
\begin{align*}
  E(y_i) = E(\epsilon_i) + E(\sum_j x_{ij}\beta_j) = \sum_jE(x_{ij}\beta_j)
\end{align*}
$x_{ij}$ and $\beta_j$ are constants which have themselves as expectation value leaves us with:
\begin{align*}
    E(y_i) = \sum_jx_{ij}\beta_j = \boldsymbol{X_i}_*\boldsymbol{\beta}
\end{align*}
To find the variance we again recognize that $\boldsymbol{X}\boldsymbol{\beta}$ follow no distribution which leaves us with only the variance of $\boldsymbol{\epsilon}$ which is given as $\sigma^2$:
\begin{align*}
  \text{Var}(\boldsymbol{y}) = \text{Var}(\boldsymbol{X}\boldsymbol{\beta} + \boldsymbol{\epsilon}) = \text{Var}(\boldsymbol{\epsilon}) = \sigma^2
\end{align*}
$\boldsymbol{\tilde{y}}$ is defined through the minimization of the mean square error $(\boldsymbol{y} - \boldsymbol{\tilde{y}})^2$ which for $\boldsymbol{\tilde{y}} = \boldsymbol{X}\boldsymbol{\beta}$ translates to the minimization of the cost function:
\begin{align*}
  C(\boldsymbol{\beta}) = \frac{1}{ n}\{(\boldsymbol{y}- \boldsymbol{X}\boldsymbol{\beta}^T\boldsymbol{y}- \boldsymbol{X}\boldsymbol{\beta})\}
\end{align*}
Where we take the derivative with respect to $\boldsymbol{\beta}$ and solve where the derivative is 0 to find the minimum:
\begin{align*}
  \frac{\partial C(\boldsymbol{\beta})}{\partial \boldsymbol{\beta}} = \boldsymbol{X}^T (\boldsymbol{y}- \boldsymbol{X}\boldsymbol{\beta}) =0
\end{align*}
\begin{align*}
  \boldsymbol{X}^T\boldsymbol{y} = \boldsymbol{X}^T \boldsymbol{X}\boldsymbol{\beta}
\end{align*}
We assume that $\boldsymbol{X}^T\boldsymbol{X}$ is invertible which gives us the the optimal $\boldsymbol{\beta}$:
\begin{align*}
  \boldsymbol{\tilde{\beta}} = (\boldsymbol{X}^T\boldsymbol{X})^{-1} \boldsymbol{X}^T\boldsymbol{y}
\end{align*}
We can now find the expectation value for the optimal $\boldsymbol{\tilde{\beta}}$:
\begin{align*}
  E(\boldsymbol{\tilde{\beta}}) &= E[(\boldsymbol{X}^T\boldsymbol{X})^{-1} \boldsymbol{X}^T\boldsymbol{y}] = E[(\boldsymbol{X}^T\boldsymbol{X})^{-1} \boldsymbol{X}^T(\boldsymbol{X}\boldsymbol{\beta} + \boldsymbol {\epsilon} )] \\
  &=  (\boldsymbol{X}^T\boldsymbol{X})^{-1} \boldsymbol{X}^TE[(\boldsymbol{X}\boldsymbol{\beta} + \boldsymbol {\epsilon} )] \\ &= (\boldsymbol{X}^T\boldsymbol{X})^{-1} \boldsymbol{X}^T\boldsymbol{X}\boldsymbol{\beta} \\&=  \boldsymbol{\beta}
\end{align*}
Here we see that the expectation value of our optimal parameter is the parameter $\boldsymbol{\beta}$.

We now find the variance of the optimal parameter:
\begin{align*}
  \text{Var}(\boldsymbol{\tilde{\beta}}) &= \text{Var}[(\boldsymbol{X}^T\boldsymbol{X})^{-1} \boldsymbol{X}^T(\boldsymbol{X}\boldsymbol{\beta} + \boldsymbol {\epsilon} )] \\
  &= \text{Var}[(\boldsymbol{X}^T\boldsymbol{X})^{-1} \boldsymbol{X}^T\boldsymbol {\epsilon} ]
\end{align*}
For a matrix $A$ we have Var$(\boldsymbol{AX}+b) = \boldsymbol{A}\text{Var}(\boldsymbol{X})\boldsymbol{A}^T$ which gives us:
\begin{align*}
  \text{Var}(\boldsymbol{\tilde{\beta}})&=(\boldsymbol{X}^T\boldsymbol{X})^{-1}\boldsymbol{X}^T\text{Var}[ {\boldsymbol{\epsilon}} ]((\boldsymbol{X}^T\boldsymbol{X})^{-1})^T\boldsymbol{X} \\  &=
  (\boldsymbol{X}^T\boldsymbol{X})^{-1}\boldsymbol{X}^T\text{Var}[ {\boldsymbol{\epsilon}} ](\boldsymbol{X}^T)^{-1}\boldsymbol{X}^{-1}\boldsymbol{X} \\&=
   (\boldsymbol{X}^T\boldsymbol{X})^{-1}\text{Var}[ {\boldsymbol{\epsilon}} ]
\end{align*}

\section{}

\end{document}
